\title{Листок 1}

\section{Решения}

\subsection{Задание 1}

Докажем, что \textbf{Poset} является категорией в соответствии с определением
\cite{awodey.category-theory/category-definition}.

\begin{enumerate}
    \item Объекты в данной категории — частично упорядоченные множества
        \textit{(из текста задачи)}.
    \item Стрелки в данной категории — монотонные функции между ними
        \textit{(из текста задачи)}.
    \item Для каждой стрелки очевидным образом определяются домен и кодомен.
    \item Имея стрелки $f : A \rightarrow B$ и $g : B → C$, хотим получить их
        композицию $g \circ f : A → C$. Она может быть определена как композиция
        соответствующих функций. Очевидно, что композиция монотонных функций
        также будет монотонной (что означает, что она также будет являться
        полноправной стрелкой в нашей категории).
    \item Для каждого объекта может быть определена единичная стрелка, которая
        преобразует множество само в себя (такая функция также будет монотонной
        из определения монотонности).
    \item Докажем выполнение свойства ассоциативности стрелок:

        $$ h \circ (g \circ f) = (h \circ g) \circ f $$

        для любых $f : A → B$, $g : B → C$, $h : C → D$. Композиция стрелок в
        нашей категории — это композиция функций, ассоциативность которой
        известна и доказывается через равенство

        $$ (h \circ (g \circ f))(a) = ((h \circ g) \circ f)(a) $$

        для любого $a \in A$.
    \item Свойство $f \circ 1_A = f = 1_B \circ f$ очевидно для композиции
        монотонной функции $f$ и единичной стрелки.
\end{enumerate}

\section{Список литературы}
% Будет добавлен автоматически при сборке.
